\chapter{Introduction}

Applications such as websearch, e-commerce, and social networking have
strict tail-latency service level objectives (SLOs) to maintain users'
attention~\cite{url:attention}.
While modern networking infrastructure allows for network transfers at
40 and 100Gbps, and round-trip latencies in a handful of
microseconds~\cite{XXX}, existing operating systems and application
software can prevent applications from achieving close to hardware
performance~\cite{ix}.
Thus, both academia and industry have focused on reducing the
overheads coming from deep software stacks and inefficient system
implementation, so that applications can perform as close to hardware
as possible.

Dataplane operating systems are such an approach that aims to optimize
throughput and latency for certain types of applications, such as
key-value stores.
Dataplane operating trade-off generality for performance, since they
can afford to simplify their networking stack, given that they serve
one specific application.
Systems such as IX~\cite{ix} and Arrakis~\cite{arrakis} showed that
re-using design principles well-known from middleboxes, can
significantly improve the application performance.

However, the above systems tried to maintain compatibility with
existing applications and only allowed minimal changes to the POSIX
API, while they maintained distinct layers between the application and
the networking stack.
This design decision, though, prevented the aggressive optimisations
that would be possible in the case of co-designing the application
with the networking stack.
For example, the way TCP is exposed to applications through POSIX
sockets, inherently implies copies between the application and the
OS.

In this thesis, we attempt to co-design and implement a key-value
store with the networking stack that depends on a reliable transport,
while at the same time completely eliminating copies both on the
receive and transmit path.
To do so, we leverage Rust, a new systems programming language, that
guarantees memory safety through an explicit memory onwernship model.
By using Rust we can reason at compile time about the memory ownership
that can change between the application and the networking stack.

We implemented a key-value store that uses the Redis API on top of
Intel's DPDK and serves requests over R2P2, a reliable transport
protocol specifically designed for RPCs.
Our key-value store depends on a lock-free store that eliminates data
copies between the application and the networking stack.
The evaluation shows that....\marios{FIXME!}

The rest of this document is organised as follows: % \marios{FIXME!}
\begin{itemize}
\item We will first explore a little background on the topic in
  chapter~\ref{chap:background}
\item We will then explain how we intend to design the system in
  chapter~\ref{chap:design}
\item After that we will see how the abstract design translates into a
  concrete Rust implementation in chapter~\ref{chap:implementation}
\item Finally we will evaluate how our system performs compared to
  other similar systems in chapter~\ref{chap:evaluation}.
\item And to conclude we will mention what further work could be done
  to improve on the current system in chapter~\ref{chap:work}
\end{itemize}
