The state of the art today for high-speed networking involves kernel
bypassing techniques making use of software such as DPDK to reduce the
overhead typically associated with kernel based networking. Such
software is usually much faster than traditional kernel networking,
mostly due to avoidance of context switches and data copies between
kernel and user space, especially when processing lots of small
packets which often happens in the case of key-value stores such as
memcached. Nonetheless these systems are far from perfect. First,
having all the  networking code and the user application inside the
same address space means that user code can corrupt the networking
stack very easily. Secondly, development of such applications is also
much more cumbersome than traditional kernel based approaches, indeed
having the user program in the same address space as the networking
code creates the need for it to be aware of low level networking
details that traditional applications don't need to worry about.The
necessity to handle this kind of networking details make kernel bypass
application much more time consuming and error-prone to develop than
regular socket based ones. Another problem these kind of systems
encounter is their inability to ensure safety, since contrary to
kernel based systems the networking stack and the user application
share a virtual address space languages like C do not provide
sufficient safety guarantees when data is shared between multiple
threads.

\marios{You need a flat abstract without subsection where you clearly
  state the problem. "We need to eliminate copies in a KV-store's
  datapath. Thus we use language features to reason about data
  onwership." You mention that we use Rust for that without any
  details. You mention that the solution is for reliable transport
  protocols and we use R2P2 without getting into details. And then you
  state some basic performance numbers.}

%%% Local Variables:
%%% mode: latex
%%% TeX-master: "master"
%%% End:
