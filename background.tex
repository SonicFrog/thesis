\chapter{Background} \label{chap:background}

We start by describing the necessary background and related work to
motivate our work.
Given that our approach aims at co-designing a key-value store with
the networking stack underneath, and do so based on specific
programming language concepts, we start by describing dataplane
operating systems, transport protocols for RPCs and then we focus on
Rust.


\section{Network Kernel-bypassing}

Today's high end NICs support speeds that make traditional kernel
networking too slow to fully exploit their capacity. The overhead of context
switches and copying is just too much when treating millions of
packets par second. Moreover the Linux kernel TCP stack is too general
for specific applications, having to handle so many quirks of TCP
makes it slower than could be achieve with a specifically designed
networking stack. To address these problems, a technique called kernel
bypassing was introduced allowing user space programs to speak
directly to the NIC through the PCIe bus thereby eliminating the
kernel overhead. These systems significantly improve throughput and
tail-latency, because they reduced the per-packet system overheads.

The advances in dataplane systems were boosted by the advent of
userspace packet processing frameworks. Intel's DPDK~\cite{dpdk} and netmap~\cite{netmap} allow userspace programs to have access to the network rings,
thus access directly the Ethernet frames, and on top implement any
necessary functionality. While netmap is interrupt based, DPDK
implements operates in a polling mode. Thus, dataplane systems use in
a tight run-to-completion loop. Every received packet runs through
network processing, then the application is called through application
callbacks, and finally the responses are sent before processing the
next packet.

Representative systems are IX~\cite{ix}, Arrakis~\cite{arrakis},
Reflex~\cite{reflex}, mOS~\cite{mos}.


Applications that have benefited from dataplane operating systems
include end-host applications with stringent latency requirements.
Key-value stores are a big part of most of today's web services
(Facebook~\cite{memcache-facebook}, LinkedIn~\cite{url:voldemort} and
Amazon~\cite{url:amazon} notably makes heavy use of key-value
stores). Applying kernel bypass techniques to key-value store could
bring huge performance improvements, by allowing a codesign of the
networking stack and the user application for our specific
purpose. MICA~\cite{mica} is an example of this, an holistic approach
to key-value stores.

However dataplane systems also come with drawbacks. They offer no
safety guarantees. Since everything usually runs in the same userspace
process, it is easy to corrupt the networking stack from the user
application. Research efforts to offer safety guarantees to dataplanes
depend either on hardware isolation and virtualization, such as IX,
or on programming language features, such as
NetBricks~\cite{netbricks}. NetBricks is a Rust framework built on top
of DPDK, aiming to provide isolated software defined network functions
while eliminating the need for virtualization. It builds upon DPDK to
provide a convenient Rust interface and provide safety guarantees
close to that of virtualization with a much lower performance
penalty.

\section{Transport protocols for RPCs}

The choice of transport protocol is critical to performance when
designing an efficient RPC protocol.
Widely used RPC protocols such as Thrift~\cite{url:thrift} and gRPC~\cite{url:grpc} depend on TCP and HTTP2 equivalently as their underlying RPC transport.

A reliable transport protocol is a necessary basis for higher-level application abstractions,
and reduces the complexity of the application code.
TCP is the defacto standard reliable transport protocol that runs on commodity hardware.
However ensuring reliability usually involves creating copies of the data sent
through the network to be able to retransmit it if needed while not
sharing data with the user program. In the mean time, these
application data could be altered. Thus copies guarantee this data integrity.
Traditional reliable networking protocols (such as the Linux kernel's
TCP stack) make heavy use of copies to avoid this problem. However
this makes the latency dependent on the size of data therefore
degrading performance, especially when the payloads grow in size. The
question is then can we provide a reliable RPC protocol while avoiding
copies altogether?

Although there are alternative reliable transport protocols both in
literature and research, all of them suffer from the redundant
copies. Such protocols are Reliable UDP~\cite{url:reludp},
QUIC~\cite{url:quic}, and R2P2.  R2P2, the Request Response Pair
Protocol, is a highly scalable RPC protocol, that provides
load-balanced delivery of multi-packet requests and responses on top
of UDP, and is designed and implemented at DCSL in EPFL.

R2P2 exposes a minimal request-response API and we decided to use that
to expose our proposed zero-copy key-value store.

\section{Rust}

Rust is a systems programming language from Mozilla Research, stemming
from Mozilla's dissatisfaction with C++. As of today several parts of
Firefox's rendering engine, Servo, have been rewritten in Rust after
the language was deemed mature enough to do so. Rust is engineered to
provide a lot of safety guarantees, notably memory and concurrency
safety through its type system thereby eliminating a lot of common
problems in systems programming.

The strong Rust guarantees also speed up development significantly by
eliminating whole categories of bugs. Most importantly, Rust's type system
eliminates use-after-free, double-free, invalid pointers, memory leaks
and data-races. Overall, the time writing code is increasing since the
compiler will often refuse straight out to compile code that
potentially violates its guarantees. The counter-part to this is that
the time spent debugging is very much reduced. This is particularly
useful when tracking down data race bugs which are notoriously hard to
diagnose. Those kind of bugs are usually found much later in the
development cycle and are usually quite hard to reproduce since they
depend on a certain scheduling of the execution.

We make the claim that Rust~\cite{rustbook} is a better suited
language than C for kernel bypass networked applications. The safety
guarantees Rust brings are very well suited for this kind of
applications, while the performance loss (if any) is negligible. As an
additional benefit Rust allows code to be much more concise and
abstract while drastically cutting down on the memory and concurrency
bug tracking time. Its memory management model will also prove
invaluable when ensuring integrity of data in between the networking
stack and the user application.

As a proof of concept we implement a key-value store on top of
DPDK in order to prove that it is doable as well as reasonable,
performance-wise, to use Rust in this kind of scenario.


\subsection{Memory safety}

One of the most important feature of Rust is its guarantee of memory
safety, providing both the speed of manual memory management and the
safety of garbage collected languages.

To provide this feature, Rust introduces a new concept called
lifetimes and makes raw pointers second class citizens. These raw
pointers (such as that of C) can't be dereferenced since the compiler
can't guarantee that they are valid. Pointers are hard to validate and
verify statically mostly because of pointer arithmetic. References, on
the other hand, are guaranteed to always point to a valid memory
location. Like the references in C++, references in Rust do not allow
pointer arithmetic, they can only be created from a variable, which
ensures that the referenced memory is valid.

Lifetimes are the central part of Rust's memory safety
guarantees. They work in conjunction with references to ensure
that references are used correctly (i.e.\ not after the value they
refer to has been destroyed), since they allow the compiler to tell
when a reference might outlive the value it refers to. The combination
of lifetimes and references thus eliminate the dangling pointers
problem entirely. Lifetimes are Rust's way of eliminating the need for
garbage collection while not requiring users to manually manage
memory. A lifetime gets attached to each variable created, the
compiler then analyzes the control flow of the program to ensure that
references or values derived from that first value (such as reference
to a struct member) are not used after the value is destroyed.

However Rust is a at the core a systems programming language so we
need some way to bypass the safety restrictions it provides. This is
where the \textbf{unsafe}~\cite{rustonomicon} keyword comes in. This
keyword allows programmers to bypass many of the safety checks of
Rust~\cite{rust-os}~\cite{redox}. It is particularly useful when
dealing with low level details, as is the case when sending packets in
a kernel bypass fashion.

\subsection{Concurrency safety}

Rust's type system also provides protection against data-races, a
common problem when using C usually solved through atomic operations
and locks. While it is possible to solve this kind of problems in C,
it requires careful design and is an error-prone task. Rust therefore
allows quicker and safer development of concurrent software. This safety
does not come for free though. In order to guarantee that there is no
data-race anywhere, Rust only allows either one mutable reference or
any number of immutable reference to any given variable. By doing
that, the compiler can ensure that the variable will only be read
concurrently or written exclusively.

Even though we speak of concurrency safety, Rust does not provide
protection against deadlocks. This is far too complex a task for
static analysis in a compiler since it requires costly call graph
analysis and sometimes produces false
positives~\cite{deadlock-detection}.

However, the Rust standard library provides safe ways to mutate shared
variables. As an example the RwLock standard library structure allows
creating mutable reference to a value from an immutable RwLock instance.

\subsection{Performance}

Being a systems programming language, performance is critical to
Rust. All the aforementioned guarantees are enforced statically so
they do not incur any performance penalty. Removing pointer aliasing
from the language also allows the compiler to do many optimizations
that are not feasible safely in C or C++. As to the performance of the
generated assembly code, Rust utilizes the LLVM compiler
infrastructure and thus benefits from all the work that was done
there. Rust also features what the Rust developers call 0-cost
abstraction. Idiomatic Rust makes heavy use of closures, and that
would incur, in some languages, a slight degradation of
performance. But, in Rust, closure can be, most of the time, inlined
thus making as if the programmer wrote the concrete code while
allowing the interface to remain abstract.

In this regard, Rust combines the best of both worlds, high-level
readable and abstract code while also having performance close to that
of other systems programming languages.

%%% Local Variables:
%%% mode: latex
%%% TeX-master: "master"
%%% End:
