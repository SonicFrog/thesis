We finish by mentioning what features are missing or could be useful
in such a piece of software. We have provided a UDP stack on top of
NetBricks, even though in real-life scenario of networking
applications UDP is rarely used. A TCP stack on top of NetBricks would
therefore be a considerable improvement.

One other thing that makes sense to have is an ARP handler. In our
case we respond to a request, so we have both the source and
destination MAC addresses at hand. But in the case where we want to
send a packet to an arbitrary IP we need to figure out the destination
MAC address. This is where an ARP protocol handler would come in
handy. \todo{IPv6}

With an ARP and TCP stack we would be close to a full featured
networking framework in Rust for kernel bypass. Such a system would
provide both the speed and the convenience of traditional socket based
networking while also providing better networking performance and the
safety of the Rust programming language. \todo{complete kernel bypass
 networking framework in Rust}

%%% Local Variables:
%%% mode: latex
%%% TeX-master: "master"
%%% End:
