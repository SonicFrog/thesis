We finish by mentioning what features are missing or could be useful
in such a piece of software. We have provided a UDP stack on top of
NetBricks, even though in real-life scenario of networking
applications UDP is rarely used. A TCP stack on top of NetBricks would
therefore be a considerable improvement.

As it stands destination MAC addresses have to be provided by the user
application when sending packets. This is less than ideal since it
means the user application still has to handle low level networking
details. In our case we respond to a request, so we have both the
source and destination MAC addresses at hand, but in some other
potential applications (such as a sending packets to an arbitrary
destination that has not sent a request) this could be a problem.

With an ARP and TCP stack we would be close to a full featured
networking framework in Rust for kernel bypass. Such a system would
provide both the speed and the convenience of traditional socket based
networking while also providing better networking performance and the
safety of the Rust programming language. A complete networking
framework for kernel bypassing in Rust, would provide an easy way of
to create safe, high throughput and low latency networking
applications in Rust.

%%% Local Variables:
%%% mode: latex
%%% TeX-master: "master"
%%% End:
